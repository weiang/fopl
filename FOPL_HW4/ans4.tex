\documentclass{paper}
\usepackage{xeCJK}
\usepackage{listings}
\usepackage{hyperref}
\usepackage{amsmath,amssymb,amsthm}
\usepackage{xcolor}

%\lstset
%{
%    basicstyle=\ttfamily,
%    numbers=left,
%    % numberstyle=\tiny,
%    keywordstyle=\color[RGB]{0, 0, 255},
%    commentstyle=\color[RGB]{0, 128, 0},
%    frame=shadowbox,
%    rulesepcolor=\color{red!20!green!20!blue!20},
%    showspaces=false,
%    showstringspaces=false,
%    extendedchars=false,
%    showtabs=false,
%    tabsize=4, breaklines,
%    xleftmargin=2em,xrightmargin=2em, aboveskip=1em
%  }

\lstset{numbers=none,
  numberstyle=\scriptsize,
  flexiblecolumns=false,
  language=Haskell,
  frame=shadowbox,
  basicstyle=\ttfamily\small,
  breaklines=true,
  extendedchars=true,
  escapechar=\%,
  texcl=true,
  showstringspaces=false,
  keywordstyle=\bfseries,
  tabsize=4}

\setCJKmainfont{AR PL UKai CN}
\title{FOPL-Homework4}
\author{昂伟 PB11011058}


\begin{document}
%	\noindent {\bf Name : 昂伟 \quad Student Id : PB11011058 \quad FOPL-Homework3}
\maketitle

\section*{Problem 1}
见{} \href{prob1.hs}{\textit{prob1.hs}}

\section*{Problem 2}
	\subsection*{(a)}
		\textit{return}与$\leftarrow$作用相反,它将一个\textbf{pure value}封装为一个\textit{IO}对象。
	
	\subsection*{(b)}
	\begin{lstlisting}
	m >>= \_ -> n
	\end{lstlisting}
	
	\subsection*{(c)}
	\begin{lstlisting}
	getLine :: IO [Char]
	getLine = getchar >>= \c -> 
                            if c == '\n'
                               then return []
                            else 
                               getLine >>= \cs -> return (c:cs)
	\end{lstlisting}
	
\section*{Problem 3}
	\subsection*{(a)}
	见{} \href{./Guesser.hs}{\textit{untilIO :: IO Bool -> IO ()}}
	\begin{lstlisting}
	untilIO :: IO Bool -> IO ()
	untilIO x = do 
	        result <- x
	        if result == True
		       then return ()
	        else 
		       untilIO(x)
	\end{lstlisting}
	
	\subsection*{(b)}
	见{} \href{./Guesser.hs}{\textit{doGuess :: Int -> IO Bool}}
	\begin{lstlisting}
	doGuess :: Int -> IO Bool
	doGuess secret = do
        guess <- getLine
        let x = read guess
        case compare x secret of 
               LT -> putStrLn "Too low!" >> return False 
               EQ -> putStrLn "Congratulations!" >> return True
               GT -> putStrLn "Too high!" >> return False

	\end{lstlisting}
	
\section*{Problem 5}
	\subsection*{(a)}
	
	\begin{minipage}{.45\textwidth}
	\centering
	\textit{Activation Records}\\
	\begin{tabular}{l|l|c}
	\hline
	(1) 		& access link 	& (0) 	\\ 	\cline{2-3}
    			& y 				& 0		\\
	\hline
	(2)		& access link 	& (1) 	\\	\cline{2-3}
   			& x				& 0	 	\\
	\hline
	(3) 	  	& access link 	& (2)		\\ 	\cline{2-3}
 			& g				& .		\\	\cline{2-3}
 			& exn handler  	& 		\\
 	\hline 
 	(4) 		&	access link & (3)	\\	\cline{2-3}
 			& 	b			& .		\\	\cline{2-3}
 			&	exn handler & 		\\
 	\hline
 	(5) 		& 	access link & (4)		\\	\cline{2-3}
 			& 	exn handler	& 19		\\	\cline{2-3}
 	\hline
 	(6) b(...)& access link 	& (4)		\\	\cline{2-3}
 			& h	 			&  .			\\	\cline{2-3}
 			& x				& 2			\\	\cline{2-3}
 			& exn handler   & 14			\\	\cline{2-3}
 			& f 				& . 			\\
 	\hline
 	(7) g(...)& access link 	& (3)		\\	\cline{2-3}
 			& h				& .		\\	\cline{2-3}
 			& x 				& 4		\\	\cline{2-3}
 			& exn handler 	& 7  	\\	
	\hline
	(8) f(...)& access link	& (6)	\\	\cline{2-3}
			& x				& 6		\\
	\hline
	\end{tabular}
\end{minipage}
\makeatletter\
\begin{minipage}{.3\textwidth}
\centering
\textit{Closures}\\
<(3), $\cdot$> \\
<(4), $\cdot$> \\
<(6), $\cdot$>
\end{minipage}
\makeatletter\
\begin{minipage}{.2\textwidth}
\centering
\textit{Compiled Codes}\\
$\textbar\ code\ for\ g \textbar $\\
$\textbar\ code\ for\ b \textbar $\\
$\textbar\ cod\ for\ f \textbar $
\end{minipage}
	\subsection*{(b)}
	活动记录(7)(8)被弹出栈。因为抛出异常时,由动态作用域知,异常处理由14行的代码处理,所以活动记录(6)之上的活动记录都被弹出。
	\subsection*{(c)}
	$y = 2$.

\section*{Problem 6}
 	\subsection*{(a)}
 	\textbf{McCarthy}定义的垃圾在我们定义中不一定是垃圾。\textbf{McCarthy}的定义中不能被基址寄存器引用的内存即为垃圾,而不能被基址寄存器引用却有可能被内存中的指针引用,所以该内存可能程序的后续指令引用,所以\textbf{McCarthy}定义的垃圾不一定是我们定义的垃圾。
 	\subsection*{(b)} 
 	我们定义的垃圾一定是\textbf{McCarthy}定义的垃圾。我们的定义中不能被程序后续指令引用的内存是垃圾,既然不能被引用当然不能被基址寄存器引用,所以我们定义的垃圾一定是\textbf{McCarthy}定义的垃圾。
 	\subsection*{(c)}
 	不可能写出一个垃圾收集器来收集完我们定义的垃圾。因为当程序执行到某点需要进行垃圾收集时,我们只能确定该点之前申请的内存空间中哪些内存一定不会被用到,但是我们无法确定程序后续指令是否一定不会用到的内存空间。所以我们只能采取相对保守的策略:只释放一定不会被用到的内存。
\end{document}